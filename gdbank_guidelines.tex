\documentclass[a4paper]{article}
\begin{document}
\title{gdbank annotation guidelines v. 0.5}
\author{Colin Batchelor}
\maketitle

\section{Background}

These guidelines assume at least a basic working knowledge of written Scottish Gaelic, particularly how to identify nouns, verbs, adjectives and determiners.

\noindent {\bf 1.} You may use a dictionary to elucidate meaning.

\noindent {\bf 2.} You may also refer to the English gloss if such a thing has been provided, for example in {\it An Litir Bheag}.

\noindent {\bf 3.} Gaelic is VSO. As a result it is very likely that the first word you come across will be either a verb or a preverbal particle.
The general pattern for preverbal particles is $\oslash$ (declarative), \textit{cha/chan} (negative), \textit{a/am/an} (interrogative), \textit{nach} (negative interrogative).





\subsection{Case}

There is no provision for marking case explicitly in either of these schemes.
However, it is useful to know about the dative and the genitive because there are dependency relations particularly for them.
For regular nouns they are marked by slenderization of the final consonant, if it can be slenderized, and lenition.
There are also irregular nouns, such as \textit{sgian} (knife), for which the genitive/dative form is \textit{sgeine}, and \textit{ball} (ball, representative) for which the genitive/dative form is \textit{bhuill}.

\subsection{Format}

\noindent {\bf 5.} gdbank follows the CoNLL-X format.
You \textbf{must} fill in the following columns: (1) token number, (2) surface, (6) FEATS, (7) head and (8) dependency type.
You \textbf{may} specify the others; if not put an underscore in them.

\section{Tokenization}

\noindent {\bf 6.} Number each of the tokens in the sentence with an integer starting at 1.

\noindent {\bf 7.} In general, break on whitespace.

\noindent {\bf 8.} Separate out commas, colons, semicolons, exclamation marks, full stops and question marks.

\noindent {\bf 9.} If a parenthesis introduces ordinary text then separate it out.

\noindent {\bf 10.} If on the other hand it indicates a list item, for example ``(3)'' or ``(iii)'', then keep it together.

\noindent {\bf 11.} Keep hyphens together with the text they join up.

\noindent {\bf 12.} Separate out \textit{dh'} and \textit{m'} and \textit{d'} but retain the apostrophe.

\noindent {\bf 13.} Also keep the apostrophe in \textit{`S} and \textit{a'} (this is important!).

\section{Dependency annotations}

\noindent {\bf 4.} If you can't work out what a particular dependency is, but you know it is there, you may use dep for completely general dependencies or mod if you know it's a modifier.

\subsection{Adjectives}

\noindent {\bf 64.} Adjectives used attributively are \texttt{amod}; predicatively are \texttt{acomp}.

\noindent {\bf 65.} Superlatives work as follows: mark \textit{nas} as \texttt{xcomp} of what it qualifies. The adjective following it is \texttt{acomp}, as it is a predicate. In the past tense, \textit{bu} is the \texttt{xcomp}, \textit{na} is related to \textit{bu} with \texttt{mark} and the adjective remains an \texttt{acomp}.

\subsection{Verbs and complements}

\noindent {\bf 14.} Mark what looks to you like the subject of a verb, even if it is the copula, with \texttt{nsubj}.

Exceptions: there are special cases.
Very often a verb will have a prepositional complement.

\noindent {\bf 15.} Mark the prepositional complement \texttt{adpmod} if it introduces an ordinary noun: \textit{Tha Mairead anns a' chidsin .}

\noindent {\bf 16.} Mark prepositional complements \textit{a'}, \textit{ag}, \textit{ri}, \textit{gu} and \textit{air} as \texttt{prt} with the verbal noun as head if they introduce a verbal noun. The verbal noun is \texttt{xcomp} with \textit{bi} as head. \textit{Tha mi a' seinn .}

\noindent {\bf 17.} Preverbal particles are marked with \texttt{prt} with the verb as the head: \textit{Chan eil .} \texttt{(prt eil Chan)} 

\subsection{Nouns and determiners}

\noindent {\bf 18.} In the case of the \texttt{det} relation the head of the NP is the head and the article is the tail. The forms of the article in Gaelic are: \textit{an}, \textit{am}, \textit{a'}, \textit{am}, \textit{na}, \textit{nam} and \textit{nan}.
\textit{gach} is also a determiner.

\noindent {\bf 36.} If a noun is modifying another noun it will typically be in the genitive, hence \texttt{(nmod Banca h-Alba)} from \textit{Banca na h-Alba}.

\noindent {\bf 41.} In the case of apposition, an NP to the right of another NP that describes it, for example in sentence gd1, ``\textit{am b\`aillidh, Patrick Sellar,}'', mark this with the \texttt{appos} relation. \texttt{(appos b\`aillidh Sellar)}. 

\noindent {\bf 42.} In the case of names, the surname is the head and the first name and, if a separate word, \textit{Mac} are both \texttt{nmod}s. \textit{an} as a separate token in a name, as in \textit{Mac an t-Saoir}, is a determiner as usual.

\noindent {\bf 43.} Treat \textit{fh\`ein} as an \texttt{amod} of the pronoun it follows.

\noindent{\bf 52.} Wh-questions are classified according to how they modify the sentence. hence \textit{Ciamar} is an \texttt{advmod}.

\noindent {\bf 73.} Heads of NPs can be \texttt{advmod} of a verb, for example \textit{oidhche} in \textit{gach oidhche}.

\subsection{Prepositions}

{\bf 19.} If the preposition takes an NP, then mark the head of that NP with \texttt{adpobj}. \texttt{(adpobj chidsin anns)} from ``\textit{anns a' chidsin}''

\noindent {\bf 20.} If the preposition is fused with a definite article, then use pobj as before but mark a reverse det dependence of the object on the fused preposition--article. Hence \textit{dhan bhaile} results in (\texttt{adpobj} dhan bhaile) and (\texttt{det} bhaile dhan).

\noindent {\bf 37.} In the case of compound prepositions, for example \textit{air s\`ailleabh}, the objects of those take \texttt{nmod} as they are in the genitive.

\noindent {\bf 61.} PPs are \textit{adpmod} of the verbs \textit{bi} and \textit{is} and equally of nouns when they are being used to express a location.

\subsection{\textit{A} and \textit{gu}}

(\textit{A} is not to be confused with \textit{a'}, two meanings of which we have already dealt with under prepositions and under articles.)

\noindent{\bf 49.} \textit{Gu}, \textit{nach} and variants introduce a dependent verb. The relation between \textit{gu} and the verb is \texttt{mark} with the verb as the head, and the verb is a \texttt{ccomp} of, in general, a noun: \textit{fios... gu}, \textit{t-eagal... gum}.

\noindent{\bf 70.} In the case where \textit{gu} has fused with a form of \textit{is}, for example \textit{gur}, then simply use \texttt{ccomp} as before.

\noindent {\bf 55.} In the phrase \textit{'S e tidsear a th' annam}, \textit{th'} is the \texttt{rcmod} of \textit{tidsear}, \textit{a} is the \texttt{rel} of \textit{th'} and \textit{annam} is an \texttt{adpmod} of \textit{th'}.

\noindent {\bf 59.} For phrases like \textit{carson a tha}, \textit{carson} and \textit{a} are both \texttt{mark} of \textit{tha}.

\noindent {\bf 62.} In the case of, say, \textit{M\`airi nach maireann} (Mary who is no longer with us), \textit{maireann} is an \texttt{xcomp} of \textit{M\`airi} and \textit{nach} is the \texttt{prt} of \textit{maireann}.

\subsection{Punctuation}

\noindent{\bf 67.} Normally, mark punctuation with \texttt{p} and the head as the verb head of the preceding phrase. In short sentences this is likely to be the root.

\noindent{\bf 78.} However, in lists, or in run-on sentences, treat commas and colons as coordination (see rule \textbf{71}).

\subsection{Coordination}

\noindent{\bf 71.} Coordination is slightly unintuitive in Stanford-based dependency schemes. The first conjunct, \textbf{not} the conjunction, is the head. Hence in \textit{ceol is gaol}, the dependency relations are \texttt{(cc ceol is)} and \texttt{(conj ceol gaol)}.

\noindent{\bf 72.} In the case of one-sided coordination, say sentences beginning with \textit{Agus} or \textit{Ach}, take the verb to be the head and make the conjunction a \texttt{cc} of the verb. (example: gd23)

\subsection{Universal Dependency Scheme dependencies by type}

\noindent {\bf 79.} For each of these dependencies I shall list ``signatures'' in terms of the universal POS tagset due to Petrov et al., which are NOUN, VERB, ADJ, ADV, PRON, DET, ADP, NUM, CONJ, PRT, \texttt{.} and \texttt{X}.

\noindent {\bf adpmod} The general relation between prepositions and either the nouns or verbs they modify. Possible signatures: (NOUN|VERB, ADP)

\noindent {\bf adpobj} The relation between the object of a preposition and the preposition. (ADP, NOUN)

\noindent {\bf det} The relation between a determiner and the noun it determines. (NOUN, DET)

\noindent {\bf nsubj} The relation between the subject of a verb and the verb. (VERB, NOUN|PRON)

\noindent {\bf dobj} The relation between the object of a verb and the verb. (VERB, NOUN|PRON)

\noindent {\bf p} The relation between a punctuation mark and the verb. (VERB, .)

\noindent {\bf ROOT} The dummy relation that identifies the head of the sentence as a whole.

\noindent {\bf mark} A relation between a verb particle and the verb. (VERB, PRT)

\noindent {\bf prt} A relation between a verb particle and the verb. (VERB, PRT)

\noindent {\bf rel} Relation between the relative particle \texttt{a} and the verb. (VERB, PRT)

\noindent {\bf ccomp} A relation between an internally-controlled clause, usually a clause that could stand on its own as a sentence, and a verb or noun. (NOUN, VERB; VERB)

\noindent {\bf xcomp} A relation between an externally-controlled clause, often a ``verbal noun'' and a verb or noun. (NOUN, VERB; VERB)

\noindent {\bf rcmod} A relation between a relative clause and, usually in Gaelic, a noun. (NOUN, VERB)

\noindent {\bf nmod} The relation between a nominal modifier (usually a noun in the genitive) and another noun. (NOUN, NOUN)

\noindent {\bf appos} The relation between a nominal modifier in apposition (usually a noun in the nominative) and another noun. (NOUN; NOUN)

\noindent {\bf amod} The relation between an attributive adjective and the noun it modifies. (NOUN, ADJ)

\noindent {\bf acomp} The relation between a predicative adjective and the verb. (VERB, ADJ)

\noindent {\bf cc} The relation between a conjunction, or comma in a list, and the first conjunct. Many possibilities. (NOUN|VERB|ADJ|ADV, CONJ|.)

\noindent {\bf conj} The relation between the first and subsequent conjunct. Many possibilities. (NOUN|VERB|ADJ|ADV, NOUN|VERB|ADJ|ADV)

\noindent {\bf advmod} The relation between an adverb and the verb it modifies. (VERB, ADV)

\section{Categories}

Reminder: these go in column 6.

\noindent{\bf 68.} If you get absolutely stuck, put X as the category.

\noindent{\bf 69.} Mark conjunctions as \texttt{conj}.

\noindent {\bf 77.} If a word has been missed out, possibly for scansion or just in error, then don't modify the types of words that would otherwise depend on it.

\subsection{Adjectives}

\noindent{\bf 28.} Most attributive adjectives are \texttt{N\textbackslash N}. \textit{\`Oran m\`or}.

\noindent{\bf 29.} Prenominal attributive adjectives are \texttt{N/N}. \textit{ath, corra, deagh, dearbh, droch, fior, s\'ar} and \textit{seann}.

\noindent{\bf 30.} Predicative adjectives are \texttt{S[adj]}.

\noindent{\bf 63.} A predicative phrase such as \textit{na bu dl\`uithe} will be \texttt{N\textbackslash N} or \texttt{S[adj]} as a whole. \textit{bu} inside will be \texttt{S[dcl]/S[adj]/N}. This means that \textit{na} is \texttt{(N\textbackslash N)/S[dcl]} or \texttt{S[adj]/S[dcl]} according to whether it is being used predicatively.

\subsection{Nouns and determiners}

Initially we won't distinguish between \texttt{N} and \texttt{NP}.

\noindent{\bf 31.} Heads of NPs are \texttt{N}.

\noindent{\bf 32.} Articles (see rule {\bf 18}), possessive pronouns (\textit{mo}, \textit{do}, \textit{a} and so forth) and \textit{gach} are \texttt{N/N}.

\noindent{\bf 33.} Nouns used attributively are \texttt{N\textbackslash N} because they follow the head.

\noindent{\bf 38.} Determiners of attributive nouns are \texttt{(N\textbackslash N)/(N\textbackslash N)}

\noindent{\bf 34.} Verbal nouns in an intransitive small clause are \texttt{S[n]}.

\noindent{\bf 35.} Verbal nouns in a transitive small clause with a pronominal object are \texttt{S[n]\textbackslash N}.

\noindent{\bf 44.} \textit{Fh\`ein} is \texttt{N\textbackslash N} unless it follows a fused preposition in which case it's \texttt{PP[*]\textbackslash PP[*]}.

\noindent{\bf 51.} Wh-question words are mostly \texttt{S[wq]/S[em]}.

\subsection{Verbs}

\noindent {\bf 21.} There are five sorts of sentence-level clauses, S[dcl] (declarative), S[q] (interrogative), S[neg] (negative), S[q] (negative interrogative) and \texttt{S[wq]} for wh-questions.

\noindent{\bf 50.} We also have \texttt{S[em]} for embedded declaratives. \textit{Ciamar a tha thu?}

\noindent {\bf 22.} Preverbal particles, for example \textit{dh'}, that indicate the type of the sentence take the type \texttt{S[*]/S[dep]}, where * is the relevant one of the foregoing. Tag the main verb with \texttt{S[dep]} and the relevant arguments.

\noindent {\bf 23.} The past-tense marker takes type \texttt{S[dep]/S[dep]} if following a preverbal particle: \textit{Cha do rinn .}
If starting a sentence the past-tense marker takes type \texttt{S[dcl]/S[dcl]}: \textit{Dh' fhaicinn .}

\noindent{\bf 40.} Intransitive clauses take the type \texttt{S[*]/N}; transitive \texttt{S[*]/N/N}.

\noindent{\bf 45.} Infinitive clauses (\textit{a} followed by a lenited verbal noun) are \texttt{S[inf]}.

\noindent{\bf 47.} \textit{Rach} introducing a passive is \texttt{S[*]/S[inf]/N}.
\noindent {\bf 58.} \textit{Bi} and variants are \texttt{S[*]/PP[*]/N}, \texttt{S[*]/S[adj]/N}, and \texttt{S[*]/PP[ri]/PP[aig]}.

\noindent{\bf 74.} For \textit{Is} constructions about psychological states, \textit{is} is \texttt{S[*]/PP[*]/N} or \texttt{S[*]/PP/S[adj]} and the psych noun or adjective will be \texttt{N/S[n]} or \texttt{S[adj]/S[n]} or \texttt{N/S[inf]} or \texttt{S[adj]/S[inf]} accordingly.

\noindent {\bf 75.} For \textit{is} constructions equating two things, use \texttt{S[*]/N/N}. 

\subsection{Prepositions}

\noindent {\bf 24.} Prepositions fused with pronouns take type \texttt{PP[*]} where * is the unfused form of the preposition.
Hence \textit{air} (meaning ``on him'') and \textit{oirre} are both \texttt{PP[air]}.

\noindent {\bf 25.} Unfused prepositions governing a non-verbal noun are \texttt{PP[*]/N}.

\noindent {\bf 26.} \textit{a'}, \textit{ag}, \textit{air} and \textit{gu} when governing a verbal noun are \texttt{PP[*]/S[n]}.

\noindent {\bf 27.} Prepositions fused with articles are \texttt{PP[*]/N} as in rule {\bf 25}.

\noindent {\bf 66.} Remember that \textit{an} can be a preposition, as in for example \textit{an d\`ochas}.

\subsection{\textit{A} and \textit{gu}}

\noindent{\bf 46.} \textit{a} before a lenited verbal noun is \texttt{S[inf]/S[n]}.

\noindent{\bf 48.} \textit{gu} and variants are usually \texttt{(N\textbackslash N)/S[dep]}, because they depend on psych nouns like \textit{fios} and \textit{eagal}.

\noindent{\bf 51.} \textit{a} in, for example, \textit{Ciamar a tha thu?} is \texttt{S[em]/S[dcl]}.

\noindent{\bf 76.} \textit{a}, the relative particle, is \texttt{(N\textbackslash N)/S[dcl]}.

\subsection{Punctuation}

\noindent{\bf 39.} Type full stops as \texttt{.}, commas as \texttt{,} and so forth.

\end{document}